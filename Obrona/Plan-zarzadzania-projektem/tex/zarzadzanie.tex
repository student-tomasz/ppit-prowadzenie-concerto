\newpage
\section{Zarządzanie}

\subsection{Cele i priorytety zarządzania}
Głównymi czynnikami motywującymi akcje zarządzania projektem są:
\begin{itemize}[nosep]
    \item nie przekroczenie budżetu projektu,
    \item uzyskanie produktów wysokiej jakości.
\end{itemize}
Dotrzymanie terminów nie jest głównym priorytetem w planie zarządzania. Jest
przewidziana swoboda z czasem przeznaczonym na pracę. Zespół wyznaczył czas
tworzenia projektu na 1 rok, przy czym harmonogram prac zakłada ukończenie
iteracji \emph{Wdrożenie} po około 5 miesiącach od rozpoczęcia projektu.

\subsection{Zarządzanie ryzykiem}
Lista zidentyfikowanych ryzyk zawarta jest w załączonym \emph{Dokumencie
zarządzania ryzykiem}.

\subsection{Mechanizmy śledzenia i kontroli}
Zadania są przydzielane zgodnie z harmonogramem prac i podziałem
odpowiedzialności. Kontrola nad postępami prac jest zapewniania przez
obowiązkowe recenzje kodu każdej zaimplementowanej części funkcjonalności
produktów. Recenzja ma miejsce przed włączeniem odpowiedzialnego kodu do
repozytorium głównego. Zarządzanie kodem źródłowym w systemie kontroli wersji
odbywa się zgodnie z metodyką \texttt{git\dywiz flow}. Gdy recenzja jest
pozytywna, zadanie jest uznawane za zakończone i kolejne zadania dodawane są do
aktualnej puli zadań oczekujących na implementację.

\subsection{Plan zatrudnienia}
Projekt jest tworzony przez siedmioosobowy zespół programistyczny:
\begin{itemize}[nosep]
    \item Tomasz Cudziło
    \item Łukasz Gwiazda
    \item Piotr Konieczny
    \item Mateusz Malicki
    \item Mateusz Ochtera
    \item Andrzej Tolarczyk
    \item Robert Wróblewski
\end{itemize}
Role przedzielone zgodnie z planem zatrudnienia z
rys.~\ref{fig:organizacja:plan_zatrudnienia} ze strony
\pageref{fig:organizacja:plan_zatrudnienia}.

W trakcie tworzenia projektu planowane jest zatrudnienie zespołu testerów
aplikacji mobilnej. Po zakończeniu iteracji \emph{Wdrożenie} możliwe jest
ograniczenie zespołu do osób niezbędnych do utrzymania pracy systemu.
Jednocześnie możliwe jest przyjęcie dodatkowej osoby koordynującej nawiązywanie
i negocjację współpracy z organizatorami wydarzeń. Pozwoli to na zwolnienie
Project Managera z części obowiązków.
