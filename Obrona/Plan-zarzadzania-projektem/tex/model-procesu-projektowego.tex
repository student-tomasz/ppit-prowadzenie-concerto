\newpage
\section{Model procesu projektowego}

\subsection{Funkcje odbioru produktów}
Projekt jest tworzony i odbierany przez ten sam zespół projektowy. Ewentualne
nadanie współpracującym organizatorom roli mającej wpływ na produkty lub cele
projektu jest rozpoznanym ryzykiem. Reakcją jest rozpatrzenie możliwości
i ewentualne wdrożenie pivotu projektu, zależnie od warunków nawiązywanej
współpracy.

Projekt realizowany jest w trzech głównych, następujących po sobie iteracjach.
Zakończenie każdej kolejnej iteracji rozszerza funkcjonalność produktów.
Planowane iteracje to:
\begin{itemize}[noitemsep]
    \item \textbf{Iteracja A} -- Przygotowanie
    \item \textbf{Iteracja B} -- Prototyp
    \item \textbf{Iteracja C} -- Wdrożenie
\end{itemize}

Zadania i produkty iteracji \emph{Przygotowanie} mają na celu stworzenie
szczegółowej dokumentacji wspomagającej i organizującej prace nad projektem. Po
zakończeniu tej iteracji rozpoczęte zostają równolegle prace nad produktami
projektu. Procesy testowania oraz integracji zachodzą możliwie ciągle.

Planowane są dwa formalne przyjęcia produktów, po zakończeniu iteracji B i C.
Akceptacja jest zadaniem zespołu projektowego. Ocena jest oparta o:
\begin{enumerate}
    \item Wytyczne dokumentu kontroli jakości w postaci zestawu testów
        akceptacyjnych;
    \item Wytyczne specyfikacji funkcjonalnej w postaci zestawu testów
        funkcjonalnych.
\end{enumerate}
Opisana metoda oceny produktów jest stałym elementem procesu testowania, który
zachodzi ciągle przez cały czas trwania prac. W efekcie zakończenie ostatniego
zadania z danej iteracji, automatycznie kończy prace nad iteracją i zostaje ona
przyjęta.

Do zakończenia iteracji \emph{Prototyp} oczekiwana jest akceptacja krytycznych
części funkcjonalności systemu. W przypadku nie uzyskania oceny pozytywnej
przewidziane jest wydłużenie czasu trwania tej iteracji. Po zakończeniu iteracji
\emph{Wdrożenie} dokonywana jest ocena pełnej funkcjonalności systemu. W
przypadku spełnienia wytycznych projekt jest uznawany za zakończony. W
przeciwnym przypadku planowane są zadania uzupełniające oraz rozszerzenie
budżetu i harmonogramu pracy.

\subsection{Synchronizacja wersji produktów}
Synchronizacja wersji produktów jest prowadzona przez system kontroli wersji
Git. Każdy produkt jest przechowywany w niezależnym repozytorium. Kiedy
wszystkie produkty dojdą do milestone'ów synchronizujących określonych w
harmonogramie pracy, ich milestone'owe rewizje zostają dodane do meta\dywiz
repozytorium całego projektu. W ten sposób prace nad produktami mogą przebiegać
niezależnie. Jednocześnie w pełni funkcjonalny produkt jest utrzymywany w
meta\dywiz repozytorium głównym.
