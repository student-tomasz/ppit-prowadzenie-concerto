\newpage
\section{Zarys projektu}

\subsection{Kontekst projektu}
Kampus Politechniki Warszawskiej jest popularnym miejscem działań agencji
marketingowych promujących wydarzenia kulturowe, w szczególności koncerty. Wybór
miejsca jest podyktowany kilkoma czynnikami:
\begin{itemize}
    \item studenci są główną grupą docelową tego typu wydarzeń,
    \item kampus zapewnia duży przepływ osób z grupy docelowej,
    \item studenci są aktywnymi konsumentami kultury --- łatwo jest ich
        zaangażować w akcjach promocyjnych.
\end{itemize}
W efekcie, nie potrzeba wiele nakładów by wypromować koncert na kampusie
Politechniki Warszawskiej. Sprawdzoną i wystarczającą metodą promocji jest
wywieszenie plakatów promujących wydarzenie w widocznych miejscach, na przykład
przed wejściami do Gmachu Głównego. Tak umieszczony plakat wystarczy, by
przekonać pojedynczych studentów. Oni natomiast zaproszą osobiście swoich
znajomych. Jedynym problemem tego scenariusza jest zawodna pamięć studentów,
zwłaszcza gdy zobaczyli plakat biegnąc na wykład, by uniknąć spóźnienia.

\subsection{Cel projektu}
Celem projektu \emph{Concerto} jest \textbf{zwiększenie efektywności plakatowych
akcji reklamowych} wydarzeń kulturalnych. W ramach projektu powstanie system
udostępniający oprogramowanie jako usługę. Usługa będzie oferowana organizatorom
wydarzeń kulturalnych.

\subsubsection{Dodatkowe cele techniczne}
Usługa \emph{Concerto} wymaga stworzenia systemu niezależnych, współpracujących
komponentów:
\begin{enumerate}
    \item aplikacji mobilnej dla adresatów akcji reklamowych,
    \item serwer obsługi zapytań od aplikacji mobilnych,
    \item API udostępniane organizatorom do zarządzania informacjami o
        reklamowanych wydarzeniach kulturalnych.
\end{enumerate}
Dokładny opis komponentów znajduje się na~stronie~\pageref{sec:produkty}.

\subsubsection{Dodatkowe cele nietechniczne}
Utrzymanie infrastruktury i zespołu projektu wymaga:
\begin{enumerate}
    \item nawiązania współpracy z organizatorami, na warunkach
        przynoszących zysk, pomimo stosunkowo wysokich kosztów utrzymania
        infrastruktury systemu;
    \item uzyskania stałej bazy użytkowników aplikacji mobilnej,
        potrzebnej przy negocjowaniu warunków współpracy z użytkownikami.
\end{enumerate}

\subsection{Założenia opłacalności}
Źródłem przychodów jest prowizja od sprzedaży biletów, które zostały dokonane z
polecenia aplikacji mobilnej \emph{Concerto}. Wysokość prowizji jest ustalana
podczas nawiązywania współpracy z organizatorami wydarzeń. Warunki współpracy
zależą od:
\begin{itemize}
    \item fazy wdrożenia projektu. Wczesna współpraca zakłada lepsze warunki
        dla organizatorów. Zespołowi projektowemu daje to możliwość uniknięcia
        konieczności zaciągnięcia kredytu.
    \item aktualnego rozmiaru bazy użytkowników aplikacji mobilnej.
\end{itemize}
Wstępnie rozwój usługi zakłada nawiązywanie współpracy z możliwie największą
grupą organizatorów wydarzeń. W przypadku napotkania problemów z uzyskaniem
zadowalających przychodów możliwa jest zmiana polityki rozwoju. Alternatywną
ścieżką rozwoju jest oferowanie współpracy tylko najważniejszym organizatorom
wydarzeń z warunkiem ekskluzywności.

\subsection{Szacowany budżet i wymagane zasoby}

\subsubsection{Zasoby techniczne}
Podczas prac planowany jest jednorazowy zakup sprzętu serwerowego i
reprezentatywnego zestawu urządzeń mobilnych potrzebnych w procesie testowania.
Budżet na zasoby techniczne na potrzeby prac nad projektem wynosi 9 000 PLN.

\subsubsection{Zasoby czasowe}
Zespół ma jeden rok na realizację projektu.

\subsubsection{Zasoby ludzkie}
Zespół projektowy składa się z:
\begin{itemize}[nosep]
    \item zespołów programistycznych -- 5 osób;
    \item manager projektu i produktu -- 1 osoba;
    \item analityk -- 1 osoba.
\end{itemize}
Wynagrodzenie zespołu, na czas tworzenia projektu, szacowane jest na 140 000
PLN.
