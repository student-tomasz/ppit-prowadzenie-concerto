\newpage
\section{Proces techniczny}

\subsection{Metody, narzędzia i techniki}
W ramach projektu zespół będzie stosował metodykę zwinną \emph{Kanban}.
Narzędzia wspomagające metodykę są częścią pakietu usług firmy \emph{Atlassian}.
\begin{enumerate}
    \item Repozytoria kodu źródłowego dla systemów kontroli wersji oraz wiki z
        dokumentami powiązanymi przetrzymywane są na platformie
        \emph{Bitbucket}.
    \item Lista zadań, śledzenie postępów oraz ewentualne dyskusje znajdują się
        na platformie \emph{GreenHopper}.
    \item Recenzje kodu odbywają się na poziomie głównego repozytorium.
        Włączenia kodu, prośby i realizacja jego recenzji odbywa się na
        \emph{Bitbucket}.
    \item Automatyczne testowanie i zarządzanie aktualną wersją oprogramowania
        serwerów produkcyjnych zarządzane są z platformy \emph{Bamboo}.
\end{enumerate}
Usługi są zintegrowane i dostępne wszystkim członkom zespołu projektowego.
Komunikacja odbywa się w ramach wymienionych narzędzi, pomiędzy wszystkimi
poziomami struktury organizacyjnej. Pozwala na to mały rozmiar zespołu
projektowego oraz bezpośrednie zaangażowanie w prace techniczne nad produktami
przez cały zespół.

\subsubsection{Serwer obsługi zapytań i API}
Do utworzenia serwera i API wykorzystany będzie stos technologiczny
składający się z:
\begin{itemize}[nosep]
    \item \texttt{Nginx, Unicorn} -- serwery, obsługa i balansowanie zapytań HTTP,
    \item \texttt{Ruby on Rails} -- framework dla aplikacji serwerowej,
    \item \texttt{PostgreSQL} -- baza danych,
    \item \texttt{RSpec, Rake, RMagick} -- biblioteki języka Ruby oferujące
        potrzebną funkcjonalność.
\end{itemize}

\subsubsection{Aplikacja mobilna}
Do stworzenia aplikacji mobilnej na system operacyjny Windows Phone 8
zostanie wykorzystany standardowy zestaw bibliotek z pakietu \texttt{.NET 4.5}.

\subsection{Dokumentacja oprogramowania}
W ramach tworzenia projektu zostaną utworzone i będą wykorzystywane dokumenty:
\begin{itemize}
    \item Szablony dokumentów -- wymuszają wspólny styl na dokumentach projektu
        oraz zapewniają obecność wymaganych elementów.
    \item Dokument stylu kodu -- do użytku wewnętrznego.
    \item Specyfikacja funkcjonalna produktów projektu -- Powstaje podczas
        iteracji \emph{Przygotowanie}. Jest podstawą do tworzenia specyfikacji
        technicznej, oraz dokładnych warunków testów akceptacyjnych i metryk z
        dokumentu kontroli jakości.
    \item Dokumentacja techniczna API -- jeden z produktów iteracji
        \emph{Wdrożenie}. Niezależny dokument, przeznaczony do użytku
        zewnętrznego przez współpracowników.
    \item Dokumentacja techniczna produktów -- tworzona jest ciągle podczas
        trwania prac nad kodem. Opisuje interfejsy klas w kodzie źródłowym.
        Przeznaczona do użytku wewnętrznego przez zespół projektowy.
    \item Specyfikacja techniczna produktów -- tworzona jest ciągle w ramach
        procesu testowania. Jest w formie wykonywalnego kodu, wchodzi w skład
        zestawu automatycznych testów funkcjonalnych.
\end{itemize}
