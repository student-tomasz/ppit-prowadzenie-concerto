\documentclass[10pt]{dokument-ppi}

\begin{document}


\Cwiczenie{2}
\Tytul{Statut projektu}
\Data{2012-12-27}
\Wersja{3.0}
\Autorzy{TC, DG, MM, MO}
\MakeDokumentMeta

\section{Uzasadnienie projektu}

Aktualnie stosowany proces kupowania biletów na wydarzenia kulturalne jest
uciążliwy i nie zmienił się znacznie w ciągu kilku ostatnich lat. Główną metodą
promocji w tym procesie są akcje plakatowe. Projekt \emph{Concerto} umożliwia
uproszczenie procesu kupowania biletu i zwiększenie efektywności akcji
plakatowych, zwłaszcza wśród głównej grupy docelowej -- studentów.


\section{Cel projektu}

Celem projektu \emph{Concerto} jest \textbf{zwiększenie efektywności plakatowych
akcji reklamowych} wydarzeń kulturalnych. W ramach projektu powstanie system
udostępniający oprogramowanie jako usługę. Usługa będzie oferowana organizatorom
wydarzeń kulturalnych.

\subsection{Podcele techniczne}

Usługa \emph{Concerto} wymaga stworzenia systemu niezależnych, współpracujących
kompenentów:
\begin{enumerate}
    \item \textbf{aplikacji mobilnej} dla adresatów akcji reklamowych,
    \item \textbf{serwer obsługi zapytań} od aplikacji mobilnych,
    \item \textbf{API udostępniane organizatorom} do zarządzania informacjami o
        reklamowanych wydarzeniach kulturalnych.
\end{enumerate}

\subsection{Podcele nietechniczne}

Utrzymanie infrastruktury i zespołu projektu wymaga:
\begin{enumerate}
    \item \textbf{nawiązania współpracy z organizatorami}, na warunkach
        przynoszących zysk, pomimo stosunkowo wysokich kosztów utrzymania
        infrastruktury systemu;
    \item \textbf{uzyskania stałej bazy użytkowników} aplikacji mobilnej,
        potrzebnej przy negocjowaniu warunków współpracy z użytkownikami.
\end{enumerate}


\section{Założenia opłacalności}

Źródłem przychodów jest prowizja od sprzedaży biletów, które zostały dokonane z
polecenia aplikacji mobilnej \emph{Concerto}. Wysokość prowizji jest ustalana
podczas nawiązywania współpracy z organizatorami wydarzeń. Warunki współpracy
zależą od:
\begin{itemize}
    \item fazy wdrożenia projektu. Wczesna współpraca zakłada lepsze warunki
        dla organizatorów. Zespołowi projektowemu daje to możliwość uniknięcia
        konieczności zaciągnięcia kredytu.
    \item aktualnego rozmiaru bazy użytkowników aplikacji mobilnej.
\end{itemize}
Wstępnie rozwój usługi zakłada nawiązywanie współpracy z możliwie największą
grupą organizatorów wydarzeń. W przypadku napotkania problemów z uzyskaniem
zadowalających przychodów możliwa jest zmiana polityki rozwoju. Alternatywną
ścieżką rozwoju jest oferowanie współpracy tylko najważniejszym organizatorom
wydarzeń z warunkiem eksluzywności.

\section{Produkty}
\label{sec:produkty}
System Concerto składa się z trzech, wymienionych jako cele techniczne, komponentów.

\subsection{Aplikacja mobilna}
Aplikacja na urządzenia mobilne kierowana jest do adresatów akcji reklamowych. Umożliwia zakup biletów na reklamowane wydarzenie. Dodatkowo pozwala na przekazanie zaproszeń na wydarzenie znajomym użytkownika. Aplikacja jest cienkim klientem, obróbka zdjęcia i znalezienie adekwatnych informacji jest przekazywane do serwera obsługi zapytań. Szacowany koszt wykonania to 42,000pln.

\subsection{Serwer obsługi zapytań}
Serwer obsługi zapytań porównuje przesłane przez użytkownika zdjęcie, ze zdjęciami znajdującymi się w bazie danych i zwraca informacje przypisane do plakatu. Jest to najbardziej złożony element systemu, co wpływa na koszt realizacji: 63,000pln.

\subsection{API dla organizatorów}
API do zarządzania bazą danych wydarzeń jest udostępniane wyłącznie współpracującym organizatorom. Pozwala na zarządzanie informacjami o wydarzeniach. Głównym zadaniem jest dodawanie plakatów wraz z opisem, datą i miejscem nadchodzących wydarzeń. Oceniamy, że koszt nie powinien przekroczyć 49,000pln.

\section{Artefakty}

\subsection{Dokumentacja techniczna serwera}
Dokumentacja techniczna serwera opisuje wewnętrzną implementację serwera obsługi zapytań. Przeznaczona jest do użytku wewnętrznego właścicieli systemu Concerto.

\subsection{Dokumentacja techniczna API}
Dokumentacja techniczna API interfejs programistyczny udostępniany ograniczonemu gronu współpracujących organizatorów.

\subsection{Dokumentacja procesu testowania}
Dokumentacja procesu testowania zawiera przebieg procesów testowania. Do użytku wewnętrznego.

\subsection{Statut projektu}
Statut projektu opisuje najważniejsze zagadnienia projektu. Opisuje cel, założenia i zakres całego projektu.

\section{Wyznaczniku sukcesu}

\begin{itemize}
  \item \textbf{Nawiązanie współpracy z organizatorami koncertów} - Za satysfakcjonujące uznamy nawiązanie współpracy z ilością organizatorów zapewniające obsługiwanie przynajmniej 50 procent koncertów w Polsce.

  \item \textbf{Uzyskanie zadowalającej liczby aktywnych użytkowników} - Pożądane jest, aby minimum 80 procent zakupów biletów u rganizatorów odbywało się za pośrednictwem aplikacji mobilnej.
  
  \item \textbf{Przychody przewyższają rozchody} - Oczekujemy, że saldo systemu będzie oscylowało miesięcznie w granicach 35,000pln.
\end{itemize}



\section{Reszta}

Reszta jest \textsc{TODO}. Generalnie trzeba skopiować te paragrafy, które
zostaną na mapie pamięci.

\newpage
\section*{Historia dokumentu}
\begin{versions}
    \version*{0.1}{2012-11-10}{MM}%
        Dodano sekcje i opisy elementów mapy pamięci.
    \version{0.2}{2012-11-10}{DG}%
        Rozszerzono sekcje \emph{Ryzyko}, \emph{Technologie} i \emph{Kontrola}.
    \version{0.3}{2012-11-10}{MM}%
        Rozszerzono sekcje \emph{Materiały}, \emph{Produkty} i \emph{Komunikacja}.
    \version{0.4}{2012-11-10}{DG}%
        Dodano propozycje rozwiązań w sekcji \emph{Ryzyka}.
    \version{0.5}{2012-11-10}{TC}%
        Dodano sekcje \emph{Uzasadnienie projektu}, \emph{Cel projektu} i
        \emph{Oczekiwane rezultaty}.
    \version{1.0.rc}{2012-11-10}{MO}%
        Sprawdzono.
    \version{1.0}{2012-11-10}{TC}%
        Zatwierdzono.
    \version{1.1}{2012-12-27}{TC}%
        Rozwinięto \emph{Cele projektu}. Usunięto sekcję \emph{Oczekiwane
        rezultaty}. Dodano \emph{Założenia opłacalności}.
\end{versions}


\end{document}
