\documentclass[10pt]{dokument-ppi}

\begin{document}


\Cwiczenie{2}
\Tytul{Statut projektu}
\Data{2012-01-03}
\Wersja{1.2}
\Autorzy{TC, DG, MM, MO}
\MakeDokumentMeta

\section{Uzasadnienie projektu}

Aktualnie stosowany proces kupowania biletów na wydarzenia kulturalne jest
uciążliwy i nie zmienił się znacznie w ciągu kilku ostatnich lat. Główną metodą
promocji w tym procesie są akcje plakatowe. Projekt \emph{Concerto} umożliwia
uproszczenie procesu kupowania biletu i zwiększenie efektywności akcji
plakatowych, zwłaszcza wśród głównej grupy docelowej -- studentów.


\section{Cel projektu}

Celem projektu \emph{Concerto} jest \textbf{zwiększenie efektywności plakatowych
akcji reklamowych} wydarzeń kulturalnych. W ramach projektu powstanie system
udostępniający oprogramowanie jako usługę. Usługa będzie oferowana organizatorom
wydarzeń kulturalnych.

\subsection{Podcele techniczne}

Usługa \emph{Concerto} wymaga stworzenia systemu niezależnych, współpracujących
kompenentów:
\begin{enumerate}
    \item \textbf{aplikacji mobilnej} dla adresatów akcji reklamowych,
    \item \textbf{serwer obsługi zapytań} od aplikacji mobilnych,
    \item \textbf{API udostępniane organizatorom} do zarządzania informacjami o
        reklamowanych wydarzeniach kulturalnych.
\end{enumerate}
Dokładny opis komponentów znajduje się na~stronie~\pageref{sec:produkty}.


\subsection{Podcele nietechniczne}

Utrzymanie infrastruktury i zespołu projektu wymaga:
\begin{enumerate}
    \item \textbf{nawiązania współpracy z organizatorami}, na warunkach
        przynoszących zysk, pomimo stosunkowo wysokich kosztów utrzymania
        infrastruktury systemu;
    \item \textbf{uzyskania stałej bazy użytkowników} aplikacji mobilnej,
        potrzebnej przy negocjowaniu warunków współpracy z użytkownikami.
\end{enumerate}


\section{Założenia opłacalności}

Źródłem przychodów jest prowizja od sprzedaży biletów, które zostały dokonane z
polecenia aplikacji mobilnej \emph{Concerto}. Wysokość prowizji jest ustalana
podczas nawiązywania współpracy z organizatorami wydarzeń. Warunki współpracy
zależą od:
\begin{itemize}
    \item fazy wdrożenia projektu. Wczesna współpraca zakłada lepsze warunki
        dla organizatorów. Zespołowi projektowemu daje to możliwość uniknięcia
        konieczności zaciągnięcia kredytu.
    \item aktualnego rozmiaru bazy użytkowników aplikacji mobilnej.
\end{itemize}
Wstępnie rozwój usługi zakłada nawiązywanie współpracy z możliwie największą
grupą organizatorów wydarzeń. W przypadku napotkania problemów z uzyskaniem
zadowalających przychodów możliwa jest zmiana polityki rozwoju. Alternatywną
ścieżką rozwoju jest oferowanie współpracy tylko najważniejszym organizatorom
wydarzeń z warunkiem eksluzywności.


\section{Produkty}
\label{sec:produkty}
System \emph{Concerto} składa się z trzech, wymienionych jako cele techniczne,
komponentów.

\subsection{Aplikacja mobilna}
Aplikacja na urządzenia mobilne kierowana jest do adresatów akcji reklamowych.
Umożliwia zakup biletów na reklamowane wydarzenie. Dodatkowo pozwala na
przekazanie zaproszeń na wydarzenie znajomym użytkownika. Aplikacja jest cienkim
klientem, obróbka zdjęcia i znalezienie adekwatnych informacji jest przekazywane
do serwera obsługi zapytań. Szacowany koszt wykonania to 42,000pln.

\subsection{Serwer obsługi zapytań}
Serwer obsługi zapytań porównuje przesłane przez użytkownika zdjęcie, ze
zdjęciami znajdującymi się w bazie danych i zwraca informacje przypisane do
odnalezionego plakatu. Jest to najbardziej złożony element systemu.
Koszt realizacji to 63,000pln.

\subsection{API dla organizatorów}
API do zarządzania bazą danych wydarzeń jest udostępniane wyłącznie
współpracującym organizatorom. Pozwala na zarządzanie informacjami o
wydarzeniach. Głównym zadaniem jest dodawanie plakatów wraz z opisem, datą i
miejscem nadchodzących wydarzeń. Oceniamy, że koszt stworzenia API nie powinien
przekroczyć 49,000pln.


\section{Artefakty}

\subsection{Dokumentacja techniczna serwera}
Dokumentacja techniczna serwera opisuje implementację serwera obsługi
zapytań. Przeznaczona jest do użytku wewnętrznego zespołu projektowego systemu
\emph{Concerto}.

\subsection{Dokumentacja techniczna API}
Dokumentacja techniczna API jest udostępniana współpracującym organizatorom.

\subsection{Dokumentacja procesu testowania}
Dokumentacja procesu testowania zawiera przebieg procesów testowania. Do użytku
wewnętrznego.

\subsection{Statut projektu}
Statut projektu opisuje najważniejsze zagadnienia projektu. Opisuje cel,
założenia i zakres całego projektu.


\section{Wyznaczniku sukcesu}

\subsection{Nawiązanie współpracy z organizatorami koncertów}
Za satysfakcjonujące uznamy nawiązanie współpracy z ilością organizatorów
zapewniające obsługiwanie przynajmniej 80\% koncertów w Warszawie i średnio 50\%
koncertów organizowanych na terenie Polski.

\subsection{Uzyskanie bazy aktywnych użytkowników}
Pożądane jest, aby minimum 80\% zakupów biletów u współpracujących organizatorów
odbywało się z polecenia aplikacji mobilnej.

\subsection{Przychody przewyższają rozchody}
Oczekujemy, że saldo systemu będzie oscylowało miesięcznie w granicach
35,000pln.


\section{Zasoby}

\subsection{Zasoby materialne}

\subsubsection{Serwer}
Serwer jest ogólnym zasobem świadczącym usługi na rzecz aplikacji mobilnej.
Przed wdrożeniem koszt utrzymania serwera jest pomijalnie niski z uwagi na
znikome obciążenie generowane przez programistów/testerów. Wdrożenie projektu
zakłada wynajęcie wirtualnego serwera od zewnętrznej firmy. Z uwagi na
przewidywane wysokie obciążenie miesięczna cena wyniesie około 4,200usd.

\subsubsection{Urządzenia mobilne}
Zespół potrzebuje platformy testowej. Reprezentatywny zespół, aktualnie
najpopularniejszych modeli telefonów komórkowych i tabletów. Służą do testowania
aplikacji mobilnej. 7000pln powinno pokryć koszt zakupu.

\subsubsection{Baza danych wydarzeń kulturalnych}
Niezbędna do działania systemu jest baza zawierająca informacje o wydarzeniach
kulturalnych takie jak ich opis, data, miejsce oraz plik z plakatem. Utrzymanie
bazy danych jest pokryte w kosztach serwera. Zawartość bazy danych jest
dostarczana przez współpracujących organizatorów.

\subsection{Zasoby czasowe}
Zespół ma jeden rok na realizację projektu.

\subsection{Zasoby ludzkie}
W zespole każda osoba ma określoną funkcję. Możemy wyróżnić:
\begin{itemize}
    \item \textbf{zespoły programistyczne} -- 5 osób; programiści podzieleni są
        na zespoły pracujące nad aplikacją mobilną, algorytmu pracującego po
        stronie serwera oraz tworzenia i obsługi serwerów udostępniających API.
        Każdy programista zarabia 40pln na godzinę.
    \item \textbf{manager projektu i produktu} -- 1 osoba; koordynuje pracę,
        odpowiada za prawidłowy przebieg projektu, oraz powstające produkty.
        Jego stawka godzinowa wynosi 60pln.
    \item \textbf{analityk} -- 1 osoba; wspomaga programistów odpowiedzialnych za serwer
        obsługi zapytań, zarobki to 40pln na godzinę.
\end{itemize}
Niezbędne są również osoby z zewnątrz tj. przedstawiciele współpracujących
organizatorów. Firmy, które nawiązały współpracę biznesową z firmą
\emph{Concerto} biorą czynny udział w tworzeniu bazy wydarzeń kulturalnych.
Elementem krytycznym jest by zachować aktualność bazy wydarzeń.


\section{Ryzyko}
Lista najistotniejszych ryzyk związanych z projektem znajduje się poniżej, zaś
pełna taksonomia ryzyk zawarta jest w oddzielnym dokumencie.

\begin{itemize}
    \item \textbf{Złożoność systemu} -- Projekt zakłada rozdzielenie systemu na
        kilka niezależnych modułów, które muszą ze sobą współpracować. Synchronizacja
        rozwoju modułów i weryfikacja ich współpracy jest nietrywialna.
    \item \textbf{Niewystarczające kwalifikacje i doświadczenie zespołu} --
        Wymagana specjalistyczna wiedza z zakresu przetwarzania obrazów. Zebranie
        zespołu posiadającego doświadczenie w tej dziedzinie jest nieopłacalne przy
        dostępnych zasobach projektu. Zespół po szkoleniu nadal może nie posiadać
        wystarczającej wiedzy.
    \item \textbf{Zbyt mała liczba współpracujących organizatorów} -- Zbyt mała
        ilość współpracujących organizatorów wydarzeń przełoży się na ubogą bazę
        materiałów referencyjnych w bazie. Dodatkowo, każdy współpracownik jest
        bezpośrednio związany z udzielanymi rabatami dla użytkowników. Zmniejszy to
        atrakcyjność oferty.
    \item \textbf{Sprawność operacyjna} -- Możliwe są problemy z efektywną
        komunikacją między podsystemami oraz z zachowaniem uptime'u systemu przy
        nieregularnych skokach obciążenia.
    \item \textbf{Poparcie biznesowe} -- Wdrożenie projektu jest niemożliwe bez
        wczesnego wsparcia finansowego inwestorów.
    \item \textbf{Przyjęcie aplikacji mobilnej przez użytkowników} -- Użytkownicy
        smartfonów mogą zignorować pojawienie się systemu \emph{Concerto}.
\end{itemize}


\newpage
\section*{Historia dokumentu}
\begin{versions}
    \version*{0.1}{2012-11-10}{MM}%
        Dodano sekcje i opisy elementów mapy pamięci.
    \version{0.2}{2012-11-10}{DG}%
        Rozszerzono sekcje \emph{Ryzyko}, \emph{Technologie} i \emph{Kontrola}.
    \version{0.3}{2012-11-10}{MM}%
        Rozszerzono sekcje \emph{Materiały}, \emph{Produkty} i \emph{Komunikacja}.
    \version{0.4}{2012-11-10}{DG}%
        Dodano propozycje rozwiązań w sekcji \emph{Ryzyka}.
    \version{0.5}{2012-11-10}{TC}%
        Dodano sekcje \emph{Uzasadnienie projektu}, \emph{Cel projektu} i
        \emph{Oczekiwane rezultaty}.
    \version{1.0.rc}{2012-11-10}{MO}%
        Sprawdzono.
    \version{1.0}{2012-11-10}{TC}%
        Zatwierdzono.
    \version{1.1}{2012-12-27}{TC}%
        Rozwinięto \emph{Cele projektu}. Usunięto sekcję \emph{Oczekiwane
        rezultaty}. Dodano \emph{Założenia opłacalności}.
    \version{1.2}{2012-01-03}{MM, MO}%
        Usunięto lub dodano paragrafy, by zachować spójność z mapą pamięci
        projektu.
\end{versions}


\end{document}
